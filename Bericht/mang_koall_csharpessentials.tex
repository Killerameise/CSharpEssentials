% Muster für die Seminarausarbeitung
% HPI Potsdam

\documentclass[11pt, a4paper]{article}

\usepackage[ngerman]{babel}
\usepackage[utf8]{inputenc} %Korrekte Kodierung der Umlaute nach UTF-8
\usepackage[T1]{fontenc} %Korrekte Kodierung der Umlaute nach UTF-8
\usepackage{amsfonts}
\usepackage{amssymb}
\usepackage{epsfig}   % Zum Einbinden von Bildern
\usepackage{url}      % Korrekter Satz von URLs
\usepackage{soulutf8}
\usepackage{color}    % Verwendung von Farben
\usepackage{listings} % Korrekter Satz von Listings und Quellcode
\usepackage[nonumberlist]{glossaries}
\usepackage[autostyle=true,german=quotes]{csquotes}
\usepackage{eurosym}

%Glossareinträge

\newglossaryentry{mvc}{name={Model-View-Controller},description={Ein Softwarepattern, das ziemlich cool ist}}
\newglossaryentry{mcg}{name={MC Gefahr},description={Ziemlich guter Rapper}}


% make glossary
\makenoidxglossaries

%Hilfs-Fonts - ohne Serifen (hier für Tabellen)
\newfont{\bib}{cmss8 scaled 1040}
%\newfont{\bibf}{cmssbx8 scaled 1040}

\definecolor{lightgray}{gray}{0.85}

%Seitenformat-Definitionen
\topmargin0mm
\textwidth147mm
\textheight214mm
\evensidemargin5mm
\oddsidemargin5mm
\footskip19mm
\parindent=0in

\begin{document}          

\begin{titlepage}
  \begin{center} 
    \mbox{}
    \vspace{1cm}
    
    {\huge C\# Essentials \\[1em] {\LARGE }}  
        
    \vspace{5cm}
    
    Seminararbeit im Seminar \\[1em]
    {\large \sc Testen, Verifizieren Analysieren von Software} \\[1em]
    Wintersemester 2015/16 \\[1em]
    Hasso-Plattner-Institut für Softwaresystemtechnik GmbH \\[1em]
    Universität Potsdam
    
    \vspace{4cm}
    
		vorgelegt von
		
    \vspace{1em}
    
		{\Large Sebastian Koall} \\
		{\Large Jaspar Mang}
		
    \vspace{4em}
    
    24.~Januar 2016
  \end{center}
\end{titlepage}


\setcounter{page}{1}

% Dritte Seite = Inhaltsverzeichnis
\tableofcontents 

\newpage

\printnoidxglossaries

\newpage

% Vierte Seite = Hier geht's eigentlich richtig los
\section{Einleitung}
C\# Essentials ist eine Erweiterung für Microsoft Visual Studio 2015. Sie wurde von Dustin Campbell, einem Developer von Microsoft und Mitentwickler von Visual Studio, programmiert.\cite{csharpEssentials} Die Erweiterung richtet sich an C\# Entwickler und zeigt dem Benutzer auf Grundlage der .NET Compiler Plattform „Roslyn“ Code-Fixes und Refactoring-Methoden. Dadurch ist es dem Benutzer einfacher neue C\# 6 Sprachfeatures zu nutzen und korrekt umzusetzen.\\
Im Rahmen der Veranstaltung \glqq Testen, Verifizieren Analysieren von Software\grqq{} haben wir die Visual Studio Erweiterung C\# Essentials auf Fehler überprüft. Dazu haben wir versucht eine möglichst Hohe Testabdeckung in verschiedenen Ebenen zu erreichen. Außerdem haben wir die Software statisch auf Fehler analysiert.\\
Beim Einrichten der Entwicklungsumgebung für C\# Essentials ist uns ein erster Defizit aufgefallen. Bei einer Konfiguration wurde von uns ein Mac in Kombination mit einer virtuellen Maschine mit Windows verwendet. Dadurch wurde bei allen Zeilenumbrüchen (Line Endings) der Mac OS Standard Zeilenvorschub (line feed) benutzt. Bei den bereits vorhandenen Tests von C\# Essentials wird mit ganzen Code Schnipseln getestet. Daher schlugen alle Tests die Code Schnipsel über mehrere Zeilen verwendeten fehl, da der andere Zeilenumbruch bei der Überprüfung zu einem Fehler führte.
 
\newpage
%
\section{Testen}

\newpage
%
\section{Statische Analyse}

\subsection{Gendarme}
Gendarme analysiert Programme und Bibliotheken, welche in den verschiedenen .NET Sprachen geschrieben wurden. Dafür benötigt Gendarme eine kompilierte Assembly, weil es nicht den C\# statisch analysiert, sondern ein daraus erzeugte Spracheformat: Das ECMA CIL.~\cite{ecma} Die CIL wird mit Hilfe von der Bibliothek Cecil analysiert. Gendarme generiert auf Basis dieser Analyse Berichte. Die Berichte 

\paragraph{ECMA CIL}
\newpage
%
\section{Diskussion der erzielten Ergebnisse}

\subsection{Darstellung der erzielten Ergebnisse}
\subsection{Diskussion der erzielten Ergebnisse}
\subsection{Ausblick auf die weitere Entwicklung des Produkts}
\newpage
%
\section{Diskussion der erzielten Ergebnisse}


%Hier kommt das Literaturverzeichnis
\newpage

\addcontentsline{toc}{section}{Literaturverzeichnis} % Zeile für das Inhaltsverzeichnis

\bibliography{bibfile}
\bibliographystyle{plain}

\end{document}
